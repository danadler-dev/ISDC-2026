\documentclass[11pt]{article}

%% -------------------------
%% Figures
%% -------------------------
\usepackage{tikz}
\usetikzlibrary{arrows.meta,positioning,fit}

%% -------------------------
%% Page layout (simple & standard)
%% -------------------------
\usepackage[margin=1in]{geometry}
\usepackage{setspace}
\onehalfspacing

%% -------------------------
%% Fonts & typography
%% -------------------------
\usepackage[T1]{fontenc}
\usepackage{lmodern}

%% -------------------------
%% Math & symbols
%% -------------------------
\usepackage{amssymb}
\usepackage{amsmath}
\usepackage{array}

%% -------------------------
%% Tables & figures
%% -------------------------
\usepackage{tabularx}
\usepackage{booktabs}
\usepackage{graphicx}
\usepackage{subcaption}

%% -------------------------
%% Algorithms (optional but fine)
%% -------------------------
\usepackage{algorithm}
\usepackage{algorithmic}

%% -------------------------
%% References (APA, same as arXiv paper)
%% -------------------------
\usepackage[style=apa,natbib=true]{biblatex}
\addbibresource{isdc2026.bib}

%% Make \cite behave like \citep (as in your arXiv paper)
\let\cite\citep

%% -------------------------
%% Hyperlinks (always appreciated)
%% -------------------------
\usepackage[hidelinks]{hyperref}

%% -------------------------
%% Title information
%% -------------------------
\title{\textbf{Evolutionary Systems Thinking}\\
\large From Equilibrium Models to Open-Ended Adaptive Dynamics}

\author{
Dan Adler\\
\small \texttt{dan@danadler.com}
}

\date{}  % ISDC papers often omit date

%% -------------------------
%% Document
%% -------------------------
\begin{document}

\maketitle

\begin{abstract}
Complex change is often described as “evolutionary” across economics, policy, and technology, yet most system dynamics models remain constrained to fixed state spaces, static transition rules, and equilibrium-seeking behavior. Such models can represent adaptation within predefined structures but struggle to capture the emergence of new structures, strategies, or categories over time. This paper argues that evolutionary dynamics should be treated as a core systems-thinking problem rather than a biological metaphor.

We introduce Stability-Driven Assembly (SDA) as a minimal, non-equilibrium framework in which stochastic interactions combined with differential persistence generate endogenous selection without genes, replication, or predefined fitness functions. In SDA, longer-lived patterns accumulate in the population, biasing future interactions and creating feedback between population composition and system dynamics. This feedback yields fitness-proportional sampling as an emergent property, realizing a natural genetic-algorithm-like process driven solely by stability.

Using SDA as a concrete example, we show why equilibrium-constrained models—even when simulated numerically—cannot exhibit open-ended evolution, and why evolutionary systems require population-dependent, non-stationary dynamics. We conclude by discussing implications for system dynamics, economics, and policy modeling, and by outlining how agent-based and AI-enabled approaches may support evolutionary system models in which structure and dynamics co-evolve.
\end{abstract}

\section{Introduction}

The term \emph{evolution} is widely used across economics, policy, technology, and organizational studies to describe long-term change. Firms are said to evolve, markets evolve, institutions evolve, and strategies evolve. Yet in most system dynamics models, the mechanisms underlying this change remain constrained to fixed state spaces, static transition rules, and equilibrium-seeking dynamics. As a result, such models capture adaptation within predefined structures but struggle to explain how new structures, strategies, or categories emerge over time. Evolution is invoked descriptively, but rarely implemented mechanistically.

This tension reflects a broader issue in scientific explanation. Idealized models often succeed by deliberately removing history, path dependence, and population effects in order to yield tractable equations. As philosophers of science have emphasized, idealization is not a flaw but a methodological tool \cite{potochnik2017idealization}. However, when the phenomena of interest are inherently historical, cumulative, and population-driven, equilibrium-based idealizations can obscure rather than illuminate causal structure. In such cases, simulation-based approaches may offer superior explanatory power—not because they are more detailed, but because they preserve the feedback between system state, population statistics, and future dynamics.

Evolutionary dynamics present a canonical example of this challenge. Biological evolution depends on variation and inheritance, but both presuppose molecular replication machinery. Long before such machinery existed, however, physical and chemical systems already exhibited large differences in persistence: some patterns survived orders of magnitude longer than others under given conditions. Across physics, chemistry, and materials science, stability disparities arise generically from energetic minima, symmetry, spatial organization, and hierarchical assembly. These disparities are not marginal; they span timescales from fleeting transient states to structures that persist for geological durations.

In population-based systems subject to continual stochastic assembly, differences in persistence necessarily bias population composition. Patterns that last longer accumulate and therefore become more likely to participate in subsequent interactions. This introduces a feedback loop: persistence shapes population statistics, population statistics shape future interactions, and those interactions reshape the distribution of persistence. Selection emerges not as an externally imposed criterion, but as a dynamical consequence of stability-weighted sampling.

In this paper, we introduce \emph{Stability-Driven Assembly} (SDA) \cite{adler_sda}, as a minimal framework that formalizes this mechanism. SDA shows that when stochastic assembly is coupled to differential persistence in an open, non-equilibrium system, the resulting dynamics necessarily implement fitness-proportional sampling. Evolutionary search emerges as a natural genetic algorithm (SDA/GA) \cite{adler2026sda_ga}, driven solely by stability rather than replication, predefined fitness functions, or fixed transition rates.

We argue that this mechanism exposes a limitation of equilibrium-constrained system dynamics models: even when simulated numerically, models with fixed state spaces cannot exhibit open-ended evolution. By contrast, evolutionary systems require that the effective state space, interaction structure, and selection pressures co-evolve with the population itself. The paper concludes by discussing implications for system dynamics, economics, and policy modeling, and by suggesting that agent-based and AI-augmented systems may provide a practical substrate for evolutionary system models capable of sustained novelty and structural emergence.

\section{The Informal Use of ``Evolution'' in Systems Discourse}

Across science, technology, economics, and policy, the term \emph{evolution} is widely used to describe long-term, structured change. Organizations evolve, technologies evolve, cultures evolve, and even languages and markets are routinely described using evolutionary language. In most cases, however, this usage is informal: the term functions as a descriptive placeholder for cumulative change rather than as a reference to a clearly specified dynamical mechanism.

To illustrate the breadth and persistence of this usage, Figure~\ref{fig:evo_trends} shows Google Trends search interest over a five-year period for several domain-specific phrases involving the word “evolution.” As shown in Figure~\ref{fig:evo_trends}(a), search interest is consistently distributed across biological, technological, cultural, linguistic, chemical, and stellar contexts, indicating that evolutionary language is neither confined to biology nor a transient metaphor. Figure~\ref{fig:evo_trends}(b) collapses the same data into five-year averages, emphasizing the comparable prevalence of non-biological uses—particularly in technology and culture. Despite this widespread and persistent usage, most formal system dynamics models do not implement mechanisms corresponding to open-ended evolutionary change; instead, evolutionary language is typically mapped onto equilibrium-seeking models with fixed state spaces and static transition structures.


\begin{figure}[h]
  \centering
  \begin{minipage}[t]{0.48\textwidth}
    \centering
    \includegraphics[width=\linewidth]{Evo-Trend-Line.png}
    \vspace{-0.5em}
    {\small (a) Time series (2021--2026)}
  \end{minipage}
  \hfill
  \begin{minipage}[t]{0.48\textwidth}
    \centering
    \includegraphics[width=\linewidth]{Evo-Trend-Bar.png}
    \vspace{-0.5em}
    {\small (b) Five-year averages}
  \end{minipage}

  \caption{Google Trends interest for domain-specific uses of the term ``evolution.'' The term is used persistently and comparably across biological, physical, and socio-technical domains, motivating the need for a general mechanistic account of evolutionary dynamics.}
  \label{fig:evo_trends}
\end{figure}

Across these domains, the term “evolution” is typically invoked without specifying the mechanisms that define biological evolution. There are no explicit genes, no genotype–phenotype distinction, no well-defined processes of mutation or recombination, and no inherited replicators that transmit information across generations. Instead, evolutionary language is used to describe gradual, cumulative change in systems whose internal representations, if any, remain implicit. While such usage is often intuitive and metaphorically effective, it leaves open the question of what, if anything, is actually being selected and how selection is implemented dynamically.

In socio-technical contexts such as technology, culture, language, and economics, evolutionary change is usually attributed to diffuse processes such as competition, learning, imitation, or adaptation to external pressures. These processes may produce path dependence and historical contingency, but they are rarely formalized in terms of population-level feedback that alters the space of possible future states. As a result, “evolution” functions as a narrative descriptor of long-term change rather than as a mechanistic account of how new structures, strategies, or categories arise endogenously within the system.

Even in formal scientific and mathematical settings, evolutionary language is frequently used analogically. Evolutionary game theory, for example, employs the terminology of strategies, fitness, and selection, yet typically assumes fixed strategy spaces and static payoff structures. The dynamics describe shifts in relative frequencies under predefined rules rather than the emergence of new strategies or representations. While such models are analytically powerful, they do not specify how the strategic space itself might evolve, nor how novel forms of behavior could arise through feedback between persistence, population composition, and generative processes.

Taken together, these usages highlight a common pattern: evolutionary language is applied broadly to systems that exhibit long-term change, yet the underlying dynamics are often constrained to equilibrium-seeking models with fixed state spaces. This gap between informal discourse and formal mechanism motivates the need for a general evolutionary systems framework capable of capturing open-ended, non-equilibrium dynamics driven by endogenous feedback rather than predefined optimization criteria.


\section{Stability-Driven Assembly as an Evolutionary Mechanism}

Stability-Driven Assembly (SDA) was previously introduced \cite{adler_sda} as a minimal non-equilibrium framework in which differences in persistence bias the accumulation of structure over time. Rather than revisiting the full theoretical development, this section focuses on the core dynamical mechanism by which persistence-weighted feedback reshapes population distributions and induces evolutionary search. The emphasis is on mechanism rather than implementation: SDA is presented as a general evolutionary process applicable across domains, independent of any specific physical or chemical substrate.

An SDA system consists of a population of patterns generated through stochastic interactions and removed through decay. Patterns may combine recursively, allowing structures of unbounded complexity to emerge, but the specific representational form (e.g., strings) is incidental. What matters is that patterns differ in stability, expressed as a characteristic persistence time. Patterns with longer lifetimes remain in the population for more generations, become more abundant, and therefore participate more frequently in subsequent interactions. Decay is modeled implicitly by removing expired patterns, while patterns with zero stability are created and eliminated within the same generation, effectively representing interactions that cannot persist under prevailing conditions. This lifetime-based asymmetry introduces selection without reproduction: persistence biases population statistics, population statistics bias future interactions, and evolutionary dynamics emerge from this feedback alone, without genes, replication, or predefined fitness functions.

Formally, an SDA system can be described by a tuple $(E, P, S, R, I)$, where $E$ denotes a finite set of base elements, $P$ the space of composite patterns formed from them, $S:P\rightarrow\mathbb{Z}^+$ a stability function assigning lifetimes to patterns, $R$ a replenishment process that maintains non-equilibrium conditions, and $I$ the number of interactions per generation. Patterns are removed upon expiration and base elements are replenished continuously, analogous to open-flow systems such as continuous stirred-tank reactors, but without assuming detailed balance or equilibrium.


\begin{algorithm}[h]
\caption{Stability-Driven Assembly (SDA): Minimal Dynamics}
\begin{algorithmic}[1]
\STATE Initialize population with base elements
\FOR{each generation}
  \STATE Remove expired patterns
  \STATE Replenish base elements to maintain non-equilibrium conditions
  \FOR{each interaction}
    \STATE Sample two patterns randomly from the current population
    \STATE Generate a new pattern via combination or recombination
    \STATE Assign lifetime based on stability and add to population
  \ENDFOR
\ENDFOR
\end{algorithmic}
\end{algorithm}


Pattern generation occurs through interactions between existing patterns. In its simplest form, SDA employs a concatenation operator, but this can be generalized to recombination with optional mutation. The resulting dynamics are captured by the following schematic loop: expired patterns are removed, base elements are replenished, parent patterns are sampled in proportion to their population frequency, and new patterns are generated and assigned lifetimes based on their stability. This minimal algorithmic structure is sufficient to implement a persistence-weighted sampling process.

\begin{figure}[h]
    \centering
    \includegraphics[width=0.6\textwidth]{SDA-Sym.png}
    \caption{Symbolic Stability-Driven Assembly (SDA/GA) loop. 
Base elements are continuously replenished while unstable motifs expire, shaping the active population through differential persistence. 
Patterns are sampled from the population and combined through interaction (concatenation in SDA or recombination in SDA/GA), generating new motifs that re-enter the population with lifetimes determined by their stability. 
The resulting feedback from \textit{stability} to \textit{persistence} to \textit{population composition} produces emergent selection without genes, replication, or an explicit fitness function.}
    \label{fig:sda-loop}
\end{figure}

Figure~\ref{fig:sda-loop} illustrates the core feedback underlying SDA dynamics. 
Replenishment and expiration define an active population whose composition is shaped by differential persistence. 
Patterns are sampled uniformly from this population for interaction, but because longer-lived patterns accumulate, they are more likely to participate over time. 
Newly generated motifs inherit lifetimes determined by their stability and re-enter the population, closing the feedback loop. 
This diagram emphasizes that effective selection arises endogenously from persistence-weighted population dynamics rather than from predefined fitness criteria or externally imposed optimization.

Crucially, the probability that a pattern participates in interactions is proportional to its frequency in the population. Since frequency is itself determined by persistence, stability implicitly defines a selection pressure. This equivalence allows SDA to be interpreted as a \emph{natural genetic algorithm} (SDA/GA): fitness-proportional selection arises endogenously from persistence differences rather than from explicit optimization criteria or reproduction operators. Variation enters through stochastic generation and recombination, while selection emerges automatically from differential lifetimes.

The resulting population dynamics can be described in terms of an evolving probability distribution \(P_t(p)\) over patterns. Two competing processes shape this distribution. A \emph{Create} process introduces novelty through stochastic pattern generation, spreading probability mass across pattern space. A complementary \emph{Persist} process removes patterns according to their lifetimes, biasing the population toward longer-lived motifs. In earlier work \cite{adler_sda}, these effects were formalized as distinct creation and persistence contributions to a recursive population update. For the present purposes, the essential point is that persistence induces a systematic bias that accumulates over time, producing path-dependent evolutionary trajectories rather than convergence to a fixed equilibrium.

To make the dynamical character of this process explicit, it is useful to consider a continuum approximation. In this limit, the evolution of the population distribution can be written in a Fokker--Planck--like form,
\begin{equation}
\frac{\partial P(x,t)}{\partial t}
=
\underbrace{- \nabla \cdot \big[ A[P](x,t)\, P(x,t) \big]}_{\textit{Persist (drift)}}
\;+\;
\underbrace{D \nabla^2 P(x,t)}_{\textit{Create (diffusion)}},
\end{equation}
where \(P(x,t)\) represents the density of motifs in a continuous feature space. The diffusion term captures the dispersive effect of stochastic pattern creation, while the drift term represents persistence-weighted bias toward regions of higher stability. Unlike classical Fokker--Planck equations, the drift field \(A[P](x,t)\) is not externally specified: it depends functionally on the evolving population distribution itself. As a result, the dynamics are nonlinear, history-dependent, and intrinsically non-equilibrium.

This distinction is central. In equilibrium or near-equilibrium models, drift and diffusion are fixed, permitting stationary distributions and linear solution techniques. In SDA, by contrast, the effective drift evolves with the population, continuously reshaping the landscape in which future dynamics unfold. This endogenous feedback prevents convergence to a fixed equilibrium and enables sustained novelty, cumulative structure formation, and entropy reduction driven by persistence rather than optimization. SDA thus provides a concrete example of how evolutionary dynamics can arise in systems models without replication, predefined fitness functions, or static state spaces.

\begin{figure}[h]
    \centering
    \includegraphics[width=0.6\textwidth]{SDA-top-down.png}
    \caption{Core feedback structure underlying Stability-Driven Assembly. 
Stochastic interactions generate patterns, while differential persistence shapes population composition. 
The evolving population distribution biases future interactions, closing a feedback loop between persistence, population statistics, and pattern generation. 
Selection emerges endogenously from this loop without genes, replication, or an externally imposed fitness function.}
    \label{fig:sda-feedback}
\end{figure}

Figure~\ref{fig:sda-feedback} summarizes the feedback mechanism that drives evolutionary dynamics in SDA. 
Patterns generated through stochastic interaction differ in persistence, and those that persist longer accumulate within the population. 
This accumulation reshapes the population distribution, which in turn biases which patterns are likely to participate in future interactions. 
The feedback loop closes causally: persistence shapes population statistics, population statistics shape future interactions, and interactions generate new patterns whose persistence further reshapes the population. 
What appears at the population level as selection is therefore not imposed externally, but emerges from the internal dynamics of persistence-weighted feedback.


\subsection{Illustrative Simulation Results}

To illustrate the dynamical consequences of stability-driven selection, we simulated a minimal symbolic SDA system under open, non-equilibrium conditions. Patterns were generated stochastically through interaction, assigned lifetimes based on stability, and removed upon expiration, while base elements were continuously replenished \cite{adler2026sda_ga}. No explicit fitness function, selection rule, or equilibrium constraint was imposed. As a baseline, we compared these dynamics to an unconstrained control in which persistence bias was removed.


\begin{figure}[h]
\centering
\includegraphics[width=0.6\textwidth]{SDA-entropy.png}
\caption{Shannon entropy of the population distribution under Stability-Driven Assembly (SDA) compared to an unconstrained control. In the absence of persistence bias, entropy remains high, reflecting broad exploration of pattern space. Under SDA dynamics, entropy decreases steadily as probability mass accumulates on long-lived motifs, demonstrating emergent selection and order in a non-equilibrium system without explicit fitness functions.}
\label{fig:sda-entropy}
\end{figure}


Figure~\ref{fig:sda-entropy} shows the evolution of Shannon entropy of the population distribution over time. In the unconstrained control, entropy remains high, reflecting a broadly uniform exploration of pattern space. In contrast, under SDA dynamics entropy decreases steadily, indicating the accumulation of probability mass on a small subset of long-lived motifs. This entropy reduction demonstrates the emergence of order and effective selection driven solely by differential persistence. Importantly, this occurs without genes, replication, or externally imposed optimization, confirming that persistence-weighted feedback is sufficient to induce evolutionary dynamics in an open, non-equilibrium system.


\section{Why Equilibrium-Constrained Models Miss Evolution}

The results above illustrate a general principle: evolutionary dynamics arise when population composition feeds back on future system behavior through differential persistence. In SDA, this feedback emerges endogenously. Patterns that persist longer accumulate in the population, biasing future interactions and reshaping the effective dynamical landscape over time. Selection is not imposed externally, nor encoded in an explicit fitness function, but emerges from persistence-weighted population dynamics. Entropy reduction and structured population skew follow as natural consequences of this feedback in an open, non-equilibrium system.

This mechanism stands in contrast to a broad class of equilibrium-constrained models commonly used in system dynamics, economics, and physical sciences. To achieve analytical tractability or numerical stability, such models typically assume fixed state spaces, constant transition rates, and well-mixed dynamics. Under these assumptions, long-term behavior is governed by convergence to a steady state determined by externally specified parameters rather than by endogenous population feedback. Once equilibrium is reached, directional change ceases by definition. While such models can describe relaxation toward equilibrium efficiently, they are structurally incapable of representing open-ended evolutionary change. The limitation is not one of simulation versus analysis, but of dynamical structure. Even when equilibrium models are simulated numerically, their transition rules remain fixed. All states participate equivalently in a Markovian flow, and no state alters the rules governing its own future reproduction or persistence. As a result, population-level history is erased rather than accumulated, and selection-like behavior cannot arise except by explicit design.

By contrast, SDA dynamics operate far from equilibrium. As shown in the continuum formulation, persistence induces a drift term that depends functionally on the evolving population distribution. The effective dynamics therefore change over time: as stable patterns accumulate, they reshape the probability landscape in which future patterns are generated. This self-consistent, population-dependent drift has no analogue in equilibrium-constrained models with fixed transition kernels. What appears as a steady state in equilibrium theory corresponds, in SDA, to a transient configuration maintained by ongoing turnover, competition, and selection.

From a systems perspective, the key distinction is whether a model allows its effective dynamics to evolve with the population it describes. Equilibrium-constrained models assume that structure is fixed and behavior adapts within that structure. Evolutionary systems, by contrast, require that structure itself be subject to selection. SDA provides a minimal example of such a mechanism, demonstrating that open-ended evolutionary dynamics can arise without genes, replication, or predefined fitness functions, but not without abandoning equilibrium constraints.

\section{Toward Open-Ended Evolutionary System Models}

The analysis above suggests that capturing evolutionary dynamics in systems models requires more than numerical simulation of fixed equations. What distinguishes evolutionary systems is not complexity per se, but the presence of feedback that allows population structure to reshape the dynamics that generate it. In such systems, the effective state space, interaction rules, and selection pressures evolve together over time. Models that hold these elements fixed, even when simulated far from equilibrium, can exhibit adaptation but not open-ended evolution.

From a systems perspective, the key requirement is endogeneity of structure. Evolutionary models must allow the entities that populate the system to influence not only aggregate variables, but also the rules governing future interactions and persistence. This implies a departure from equilibrium-constrained frameworks toward models in which transition kernels, payoff structures, or interaction networks are functions of the evolving population itself. SDA provides a minimal example of such dynamics: persistence biases population composition, population composition biases future interactions, and the resulting feedback reshapes the effective dynamical landscape.

Agent-based modeling offers a natural substrate for extending these ideas. Agents can represent patterns, strategies, or assemblies whose lifetimes and interactions depend on their internal structure and environmental context. When agents are embedded in open, replenished environments and allowed to interact stochastically, differential persistence alone can generate selection-like dynamics without explicit optimization objectives. Importantly, agent-based models can support expanding state spaces, enabling the emergence of novel behaviors or representations not anticipated at model initialization.

Recent advances in artificial intelligence further expand the design space for evolutionary systems models. Learning agents can adapt their internal representations or interaction strategies based on experience, effectively altering the space of possible future behaviors. When combined with persistence-driven selection at the population level, such systems may support genuinely open-ended evolutionary learning, in which new strategies, categories, or organizational forms arise without predefined targets. In this context, AI serves not as an optimizer but as a flexible substrate for variation and interaction.

The challenge, however, is not simply to add more learning or complexity, but to preserve the core evolutionary feedback identified in SDA. Without mechanisms that couple persistence to population-level bias, learning agents risk converging to fixed equilibria or cycling within predefined solution spaces. Open-ended evolution requires that success alters the future search process itself. Systems models that satisfy this criterion blur the traditional boundary between dynamics and structure, and between micro-level interactions and macro-level organization.

Viewed in this light, evolutionary systems thinking represents an extension rather than a rejection of system dynamics. It retains the emphasis on feedback, accumulation, and nonlinearity, but shifts attention from equilibrium behavior within fixed structures to the evolution of the structures themselves. Developing such models remains an open research frontier, with implications for economics, policy design, artificial intelligence, and the study of complex adaptive systems.

\section{Why Agent-Based Simulation Alone Is Not Sufficient}

Agent-based models are often proposed as a remedy to the limitations of equilibrium-constrained systems, since they allow heterogeneous agents, nonlinear interactions, and rich emergent behavior. While agent-based simulation is an essential ingredient of evolutionary modeling, it is not sufficient on its own. Many agent-based systems remain structurally incapable of evolution because they treat population-level structure as an output rather than as a causal state.

In typical agent-based models, agents interact according to fixed rules within a predefined state space. Population statistics—such as frequencies of strategies or behaviors—are computed as summaries of agent states but do not feed back to alter the rules governing future interactions. As a result, although such models can exhibit adaptation, learning, or complex transient dynamics, they lack a mechanism by which population composition reshapes the effective dynamics of the system itself.

The SDA framework clarifies why this distinction is critical. In SDA, the population distribution is not merely an aggregate outcome but an active dynamical variable. Differential persistence reshapes population composition, and this composition directly biases future interactions. Agents and populations are therefore dynamically inseparable: the behavior of individual entities depends on the population they inhabit, while the population evolves as a consequence of individual persistence. Evolution arises from this bidirectional coupling, rather than from agent-level optimization or externally imposed selection.

From a modeling perspective, this implies that agents cannot be simulated independently of population dynamics. Any model in which agent behavior is conditionally independent of population composition—except through fixed, externally specified parameters—will fail to capture persistence-driven evolutionary feedback. Evolutionary systems require models in which population structure mediates interaction probabilities, survival, and the generation of future variants.

This perspective helps explain why many agent-based and simulation-based approaches produce rich transient behavior yet converge to fixed regimes or repeating cycles. Without explicit population-level feedback that alters the effective transition structure, such systems can adapt within a given space of possibilities, but they cannot generate new ones. SDA provides a minimal example of how incorporating population-dependent feedback transforms simulation into genuine evolutionary dynamics.


\section{Implications for Economics and Policy Modeling}

\subsubsection{Illustrative Example: Industry Ecosystems}

The SDA framework can be interpreted in economic contexts as a model of evolving industry ecosystems. Consider a simplified system with manufacturers (A), retailers (B), and logistics providers (C) as base elements. Through interaction and coordination, these entities form composite organizational structures such as manufacturer--retailer partnerships (AB) or integrated supply networks (ABBC). Different configurations exhibit different degrees of stability, reflecting factors such as profitability, resilience, and institutional durability.

\begin{figure}[h]
    \centering
    \includegraphics[width=0.85\textwidth]{SDA-econ.png}
    \caption{Conceptual illustration of an industry ecosystem modeled as a Stability-Driven Assembly (SDA) process. 
Base economic entities (e.g., manufacturers, retailers, logistics providers) interact to form composite organizational structures. 
Configurations that persist longer under competitive and institutional pressures accumulate resources and participation, biasing future interactions. 
Over time, differential persistence shapes the population of organizational forms, producing evolutionary dynamics without centralized optimization or explicit replication.}
    \label{fig:sda-econ}
\end{figure}

More stable configurations persist longer and therefore accumulate resources, attention, and participation, increasing their likelihood of engaging in further interactions. Less viable arrangements dissolve more quickly, redistributing resources back into the system. Over time, this persistence bias skews the population toward durable industry structures, which function as attractors without requiring centralized optimization or ex ante design.

This example illustrates how stability-driven selection can operate in socio-economic systems through feedback between persistence, population composition, and future interactions. From a systems perspective, the key driver is not rational optimization at the firm level, but population-level feedback that amplifies durable organizational forms over time.

\subsubsection{Implications for Economic and Policy Systems}

Economic and policy systems are routinely described as evolutionary: firms adapt, technologies diffuse, institutions evolve, and strategies compete over time. Yet most formal models in economics and policy analysis remain equilibrium-centered or rely on fixed behavioral rules, even when implemented in dynamic or agent-based form. As a result, such models can represent adjustment within predefined structures but struggle to capture the emergence of new strategies, institutional forms, or categories of behavior.

The SDA framework suggests a different modeling emphasis. Rather than specifying selection criteria or optimization objectives ex ante, evolutionary change can arise from differential persistence under continual turnover. In economic contexts, persistence may correspond to the durability of business models, regulatory arrangements, norms, or technologies under changing conditions. Entities that persist longer naturally accumulate influence, visibility, or market share, biasing future interactions and shaping the environment in which new variants arise.

This perspective clarifies key limitations of equilibrium-based policy analysis. Policies are often evaluated by comparing steady states before and after an intervention, implicitly assuming that system structure remains fixed. In practice, however, policy interventions frequently alter the space of viable strategies and institutions, triggering long-term evolutionary effects that are invisible to equilibrium comparisons. Persistence-driven dynamics imply that small changes affecting survival or turnover rates can have outsized consequences by reshaping population composition and future trajectories.

Incorporating evolutionary feedback into economic and policy models does not require abandoning system dynamics principles, but it does require extending them. Models must allow population composition to influence future dynamics endogenously, rather than treating behavior and structure as static. Hybrid approaches that combine stock--flow models with agent-based or population-based representations offer a promising path forward, in which stocks represent populations of strategies or institutions, flows represent entry and exit governed by persistence, and feedback loops capture how accumulated structures bias future behavior.

Finally, the SDA perspective highlights a shift in policy design from optimization to resilience. Rather than attempting to engineer optimal outcomes in complex adaptive systems, policy may be better understood as shaping persistence landscapes—altering which behaviors, institutions, or technologies are able to survive and propagate. From this standpoint, effective policy is less about directing outcomes and more about creating conditions under which desirable structures can persist and evolve over time.



\section{Conclusion}

This paper has argued that many systems described as evolutionary fail to evolve in a formal sense because the models used to represent them are structurally constrained toward equilibrium. When state spaces, transition rules, and selection criteria are fixed in advance, long-term change can be simulated but not generated. Evolution, understood as cumulative, path-dependent transformation of both populations and the structures that shape them, requires feedback that allows persistence to bias future dynamics.

Stability-Driven Assembly provides a minimal demonstration of such feedback. By coupling stochastic generation with differential persistence in an open, non-equilibrium system, SDA shows how selection-like behavior can emerge without genes, replication, or predefined fitness functions. Persistence reshapes population composition, population composition biases future interactions, and this feedback produces sustained entropy reduction and structured population-level order. Interpreted through a dynamical-systems lens, SDA corresponds to a nonlinear, population-dependent drift process that lies outside the scope of equilibrium-constrained models.

The broader implication is methodological rather than domain-specific. Evolutionary systems thinking requires models in which structure itself is endogenous and subject to selection. This perspective extends system dynamics beyond equilibrium analysis toward frameworks that integrate population feedback, expanding state spaces, and non-stationary dynamics. Agent-based and AI-enabled models offer promising substrates for such work, but only insofar as they preserve the core persistence-driven feedback identified here.

Viewed in this way, evolution is not a metaphor imported from biology, but a general class of dynamical behavior that arises under specific structural conditions. Identifying and modeling those conditions—rather than assuming equilibrium or optimization—may be essential for understanding complex adaptive systems in economics, policy, and beyond.

\printbibliography

\end{document}

